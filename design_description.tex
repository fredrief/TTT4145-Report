% !TEX root = main.tex
\section{Design Description}
\label{sec:design_description}
Figure \ref{fig:block_toplevel} shows how the transmitter and receiver communicates within the system. Block diagrams for the two subsystems is shown in appendix \ref{a:block_diagram}, figure \ref{fig:block_diagram} and \ref{fig:block_diagram_feedback}. The behaviour will be explained in this section. 

Because two different modulation schemes are used, the receiver need to know how to decode the incoming data packets. This problem is solved by using two different training sequences, both of length \barkerSymbols symbols. The training sequence is an appropriate repetition of barker sequences of length 7 and 13, for QPSK and QAM-64 modulation respectively. As shown in figure \ref{fig:block_diagram}, after frame sync, the receiver perform a check on which barker sequence that was transmitted before de-mapping the symbols. 

In the transmitter, a variable called \textit{Session State} keep the information about what data quality and modulation scheme to use. As figure \ref{fig:block_diagram} indicates, the session state influences several blocks of the TX side in the transmitter. As shown in figure \ref{fig:block_diagram_feedback}, the decision of when to change data quality is left completely to the transmitter. For every received data package, the receiver computes the number of detected errors and transmit this number back to the transmitter. As the figure indicates, these BER packages are always transmitted using QPSK-modulation and no FEC is used. Based on the received number of detected errors as well as previous error detections, the transmitter decides whether to change session state or not. 

In the following subsections, all components of the data path and BER path systems will be described in detail.   
\subsection{Sound Producer and Sound Consumer}
The blocks sound producer and sound consumer contains functionality for handling the sound input and output to the sound card of the computer. Sound producer reads sound samples at from the sound card at full quality, i.e. 16 bit, 44100 Hz stereo, and writes the samples to a queue accessible for the source encoder. Sound consumer equivalently reads sound samples from a queue controlled by the unpacking block and writes to the computer sound card.

\subsection{Source Encoder/Decoder}
The source encoder performs lossy compression of the produced sound samples. The samples produced by sound producer are stored as 64 bit unsigned integers (u64) even though the resolution is only 16 bits. This means that the 48 least significant bits of each sample is zero. Depending on the state of the system (High or low quality transmission) the source encoder read several u64s and pack them into one single u64. Five or eight samples are packed into each u64 depending on the state (see details in table \ref{tab:specs_data}). In addition the sample rate are reduced by decimation with a factor of 2 or 4 for high and low quality respectively. 

The source decoder performs the inverse operation, writing 2 or 4 copies of the same sample as u64's to the sound consumer.  

\subsection{Packing}
Add header and write to packet queue. Need more info.

\subsection{Forward Error Correction}
The implemented FEC algorithm is Hamming (7,4). The implementation is a fast, pre-written C-code, written by Michael Dipperstein \cite{hamming}. 

\subsection{Symbol Mapping}
The system uses Grey Code for mapping the binary data to a complex vector $z$. For a two bit sequence $b_1b_0$, with $b_0$ as the least significant bit, the QPSK mapping is obtained by equation \ref{eq:symbol_map_qpsk1} and \ref{eq:symbol_map_qpsk2}. 
\begin{align}
\Re[z_{qpsk}] = (-2b_0+1) \label{eq:symbol_map_qpsk1} \\ 
\Im[z_{qpsk}] = (-2b_1+1) \label{eq:symbol_map_qpsk2}
\end{align}

For a 6 bit sequence $b_5b_4b_3b_2b_1b_0$ the mapping to QAM-64 is obtained by equation \ref{eq:symbol_map_qam1} and \ref{eq:symbol_map_qam2}
\begin{align}
\Re[z_{qam64}] = (2b_5-1) \cdot [(-2b_3+1) \cdot (-2b_5+3) +4] \label{eq:symbol_map_qam1} \\ 
\Im[z_{qam64}] = (2b_6-1) \cdot [(-2b_4+1) \cdot (-2b_6+3) +4]\label{eq:symbol_map_qam2}
\end{align}

The resulting constellations from these mappings are shown in figure \ref{fig:qpsk_mapping} and \ref{fig:qam_mapping} in appendix \ref{a:symbol_mapping}.

\subsection{Training sequence}
The training sequence is added to the transmitted packet in the block \textit{Add Barker} on figure \ref{fig:block_diagram}. As previously described, the training sequence is a barker sequence of length 7 or 13 depending on the session state. In either case the total length of the training sequence is \barkerSymbols symbols, which means that the barker sequence is repeated to obtain the desired length. 

\subsection{Pulse shaping}
The pulse shaping filter is applied in the last step before transmission, and is of the type Root Raised Cosine, with a Roll-off factor of 0.3. The same filter is applied as a matched filter in the first step after reception at the receive side. 

\subsection{Symbol synchronisation}
Pulse align. Need info.

\subsection{Frequency synchronisation}
The implemented frequency synchronisation is a combination two algorithms; blind frequency search and the Mth-power algorithm. The blind frequency search algorithm is applied before frame synchronisation and provides a rough estimate of frequency offset. The purpose of this block is to ensure that the phase drift of the training symbols is small enough before frame synchronisation. The blind frequency search is done by applying a fixed set of frequency offsets to the input symbols and then computing the cross correlation between the input symbols and the two training sequences. The frequency offset yielding the highest value for the cross correlation gives the best estimate for the frequency offset. This way, the remaining frequency offset for the frame sync is ensured to be within an interval small enough for the frame synchronisation algorithm to work. 

The Mth-power algorithm is applied after frame synchronisation and does both frequency and phase offset estimation. The algorithm operates on the training sequence only. The algorithm is based on mapping all symbols to the same point in the complex plane by raising each symbol to the 4th power. With out any phase or frequency shift, each symbol of the training sequence should be mapped to the point(-1, 0). The phase estimate of the acquired symbol is then $1/4$th of the angular deviation from this point. This is illustrated in figure \ref{fig:mth_pow}. The frequency offset is estimated by tracking deviation in phase. 
% !TEX root = main.tex
\begin{figure}[htbp]
\centering
\begin{tikzpicture}
% Grids
\coordinate (origo) at (0,0);
\coordinate (origo_w) at (5.1,0);

\draw[step=0.3cm,gray,very thin] (-1.51,-1.51) grid (1.5,1.5);
\draw[step=0.3cm,gray,very thin] (3.6,-1.51) grid (6.6, 1.5);

% Axis
\draw[thick, ->] (-1.5,0) -- (1.5, 0) node [anchor=north west]{\Re[$z$]};
\draw[thick, ->] (0,-1.5) -- (0,1.5) node [anchor=south west]{\Im[$z$]};
\draw[thick, ->] (3.6,0) -- (6.6, 0) node [anchor=north west]{\Re[$w$]};
\draw[thick, ->] (5.1,-1.5) -- (5.1,1.5) node [anchor=south west]{\Im[$w$]};

\draw[thick] (0.9, 0) ++(0, 2pt) -- ++(0,-4pt) node[font=\scriptsize, yshift=-0.1cm, xshift=-0.1cm]{1};
\draw[thick] (-0.9, 0) ++(0, 2pt) -- ++(0,-4pt) node[font=\scriptsize, yshift=-0.1cm, xshift=-0.1cm]{-1};
\draw[thick] (0, 0.9) ++(2pt, 0) -- ++(-4pt, 0) node[font=\scriptsize, yshift=-0.1cm, xshift=-0.1cm]{1};
\draw[thick] (0, -0.9) ++(2pt, 0) -- ++(-4pt, 0) node[font=\scriptsize, yshift=-0.1cm, xshift=-0.1cm]{-1};

\draw[thick] (6, 0) ++(0, 2pt) -- ++(0,-4pt) node[font=\scriptsize, yshift=-0.1cm, xshift=-0.1cm]{4};
\draw[thick] (4.2, 0) ++(0, 2pt) -- ++(0,-4pt) node[font=\scriptsize, yshift=-0.1cm, xshift=-0.1cm]{-4};
\draw[thick] (5.1, 0.9) ++(2pt, 0) -- ++(-4pt, 0) node[font=\scriptsize, yshift=-0.1cm, xshift=-0.1cm]{4};
\draw[thick] (5.1, -0.9) ++(2pt, 0) -- ++(-4pt, 0) node[font=\scriptsize, yshift=-0.1cm, xshift=-0.1cm]{-4};

% Plot
\draw[blue,fill=blue] (0.9,0.9) circle (2pt) coordinate[](sample);
\draw[blue,fill=blue] (-0.9,0.9) circle (2pt);
\draw[blue,fill=blue] (0.9,-0.9) circle (2pt);
\draw[blue,fill=blue] (-0.9,-0.9) circle (2pt);
\draw[red,fill=red] (0.73,1.043) circle (2pt) coordinate[](sample_err);

\draw[thin, blue] (0,0) -- (sample);
\draw[thin, red] (0,0) -- (sample_err);
\pic [draw, red, ->, angle eccentricity=1.5] {angle = sample--origo--sample_err};
\draw (0.3, 0.3) node[anchor=north west, text=red, yshift=0.2cm]{$\phi$};

\draw[blue,fill=blue] (3.83,0) circle (2pt) coordinate[](sample_w);
\draw[red,fill=red] (4.125,-0.818) circle (2pt) coordinate[](sample_err_w);
\draw[thin, blue] (origo_w) -- (sample_w);
\draw[thin, red] (origo_w) -- (sample_err_w);
\pic [draw, red, ->, angle eccentricity=1.5] {angle = sample_w--origo_w--sample_err_w};
\draw (origo_w) ++(-1.1,-0.35) node[anchor=north west, text=red, yshift=0.2cm]{$4\phi$};

% Other
\draw[thick, ->] (2, 1) .. controls (2.5,1.1) .. (3,1) node[above, anchor=south east, xshift=0.1cm]{$w=z^4$} ;

\end{tikzpicture}
\caption{Illustration the mapping $w=z^4$ which indicates how a phase error in QPSK modulated signal can be observed.}
\label{fig:mth_pow}
\end{figure}
The implemented Mth-power algorithm is executed in four stages. 
\begin{enumerate}[a)]
\item
Estimate frequency offset by calculating the average phase difference between two adjacent symbols. 

\item
Estimate phase offset by first subtracting the estimated frequency offset from each sample, then calculate the average phase offset.

\item
The above phase estimate is limited to $[0, \frac{\pi}{2}]$, so the phase offset need to be moved to the correct quadrant. Because the first symbol in the training sequence is always $1+i$, an angle of $\pi/2$ is added to the the phase estimate until the first symbol is mapped to the first quadrant. 

\item
Apply the estimated phase and frequency to all remaining symbols. 
\end{enumerate}

\subsection{Frame synchronisation}
Frame synchronisation is done by computing the crosscorrelation between the two training sequences and the received symbols. Value of the crosscorrelation is compared to a pre set threshold of 13. The first index that gives a value exceeding this threshold is considered to be the beginning of the frame. 
NEED MORE INPUT





