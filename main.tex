\documentclass[9pt,journal]{IEEEtran}
\usepackage[utf8]{inputenc}
\usepackage{cite}
\usepackage[hidelinks]{hyperref}
\usepackage[shortlabels]{enumitem}

% Math
\usepackage{amsmath}
\DeclareMathOperator\erfc{erfc}
\usepackage{nicefrac}
% For tables
\usepackage[table]{xcolor}
\usepackage{tabularx, booktabs}
\usepackage{multirow}

% Units
\usepackage{siunitx}
\sisetup{
    exponent-product = \cdot,
    separate-uncertainty = true,
    per-mode = symbol,
    group-digits = false,
    detect-weight=true, 
    detect-family=true,
    detect-all
}
% Tikz
\usepackage[europeanresistors,americaninductors, americancurrents, american voltages, siunitx]{circuitikz}
\usepackage{tikz}
\usetikzlibrary{shapes,arrows, calc, automata, positioning, chains, decorations.markings, calc, patterns, angles, quotes}
\usepackage{pgfplots}

\usepackage{graphicx} % Figures/graphics
\usepackage{float} % Floats, H
% Subfigures:
\ifCLASSOPTIONcompsoc
    \usepackage[caption=false, font=normalsize, labelfont=sf, textfont=sf]{subfig}
\else
\usepackage[caption=false, font=footnotesize]{subfig}
\fi

% New commands
\newcommand{\ebnot}{$\nicefrac{E_b}{N_0}$}
\renewcommand{\Re}{\text{Re}}
\renewcommand{\Im}{\text{Im}}

% TODOs
\newcommand\todo[1]{\textcolor{red}{#1}}


% -------------------------------- System variables ----------------------------- %
% Specs

\newcommand{\MainCarrierFreq}{2415} %MHz
\newcommand{\paPower}{-10} % dBm
\newcommand{\BERThreshold}{$1\cdot10^{-2}$} % For changing quality
\newcommand{\BERPowerLimit}{-17} % dBm

\newcommand{\bitPerSymbQPSK}{2}
\newcommand{\bitPerSymbQAM}{4}
\newcommand{\soundSampleRateQPSK}{11025}
\newcommand{\soundSampleRateQAM}{22050}
\newcommand{\bitPerSoundSample}{12}


\newcommand{\headerBits}{8}
\newcommand{\packetDataSymbolsQPSK}{456}
\newcommand{\packetDataSymbolsQAM}{228}
\newcommand{\packetDataBitsQPSK}{512}
\newcommand{\packetDataBitsQAM}{512}

\newcommand{\barkerSymbols}{26}
\newcommand{\barkerBitsQPSK}{\the\numexpr2*\barkerSymbols\relax}
\newcommand{\barkerBitsQAM}{\the\numexpr4*\barkerSymbols\relax}

\newcommand{\guardSymbols}{2}
\newcommand{\guardBitsQPSK}{\the\numexpr\bitPerSymbQPSK*\guardSymbols\relax}
\newcommand{\guardBitsQAM}{\the\numexpr\bitPerSymbQAM*\guardSymbols\relax}
\newcommand{\burstLengthSymbolsQPSK}{484}
\newcommand{\burstLengthSymbolsQAM}{256}
\newcommand{\burstSizeBitsQPSK}{331}
\newcommand{\burstSizeBitsQAM}{935}

\newcommand{\systemBitRateQPSK}{187.39} % kbits/s
\newcommand{\systemBitRateQAM}{572.67} % kbits/s
\newcommand{\symbolRateQPSK}{150 } % ksymb/s
\newcommand{\symbolRateQAM}{\symbolRateQPSK} % ksymb/s
\newcommand{\filterRollOff}{0.5}
\newcommand{\sps}{8}


\newcommand{\minBWQPSK}{112,5}% kHz
\newcommand{\minBWQAM}{\minBWQPSK}% kHz
\newcommand{\maxSystemBW}{100} % kHz

% Data rates
\newcommand{\rawDataRate}{1.4} %Mb/s
\newcommand{\sourceDataRateQPSK}{\the\numexpr\bitPerSoundSample*\soundSampleRateQPSK/1000\relax} % kb/s
\newcommand{\sourceDataRateQAM}{\the\numexpr\bitPerSoundSample*\soundSampleRateQAM/1000\relax} % kb/s
\newcommand{\packetDataRateQPSK}{134.4} % kb/s
\newcommand{\packetDataRateQAM}{268.7} % kb/s
\newcommand{\fecDataRateQPSK}{235.1} % kb/s
\newcommand{\fecDataRateQAM}{470.3} % kb/s
\newcommand{\symbolMapRateQPSK}{117.6} % ksymb/s
\newcommand{\symbolMapRateQAM}{117.6} % ksymb/s
\newcommand{\barkerRateQPSK}{124.5} % ksymb/s
\newcommand{\barkerRateQAM}{130.2} % ksymb/s
\newcommand{\sampleRateQPSK}{1.00} % M sample/s
\newcommand{\sampleRateQAM}{1.06} % M sample/s

\newcommand{\USRPSampleRate}{1.2} %Msamp/s


% BER Feedback path
\newcommand{\BERDataBits}{26}
\newcommand{\BERBurstSize}{43} % Symbols

\newcommand{\BERCarrierFreq}{2455} %MHz
\newcommand{\BERSymbolRate}{\symbolRateQPSK} %ksymb/s
\newcommand{\BERminBW}{\minBWQPSK}


% ------------------------ MEASUREMENTS -----------------------------%
\newcommand{\measBW}{220.6} %kHz, -40dBc
\newcommand{\measPWR}{-10} %dBm
\newcommand{\measPWRbad}{-25} %dBm
\newcommand{\measDelay}{262} %ms
\newcommand{\measDelayStd}{63} %ms
\newcommand{\measDelaySwitch}{40} %ms

\newcommand{\measSNRQPSKGood}{18.1} %dB
\newcommand{\measSNRQAMGood}{12.8} %dB
\newcommand{\measSNRQPSKBad}{15.9} %dB
\newcommand{\measSNRQAMBad}{10.16} %dB
\newcommand{\measEVMGood}{-9} %dB
\newcommand{\measEVMBad}{-4} %dB
\newcommand{\measBERQPSKGood}{\SI{1.35e-6}{\;}} 
\newcommand{\measBERQAMGood}{\SI{6.0e-2}{\;}} 
\newcommand{\measBERQPSKBad}{\SI{1.4e-5}{\;}} 
\newcommand{\measBERQAMBad}{\SI{1.32e-1}{\;}} 


% ------------------------ Document Variables ---------------------------- % 
\newcommand{\figW}{0.7}



% ---------------------------------- DOCUMENT ------------------------------------------------- %

\title{Demonstration of Adaptable Quality Radio System for Broadcasting of Speech}
\author{Martin Lima and Fredrik Esp Feyling }
\date{April 2020}

\begin{document}

\maketitle
\begin{abstract} 
This paper is presenting the design and implementation of a radio communication system for broadcasting of speech with adaptable data rate. This system is to be seen as a "proof of concept", where the main goal is to demonstrate a radio system with feedback from receiver (RX) to transmitter (TX) such that the transmitted data rate adapts to the state of the radio channel. The data rate is varied by a factor 2 by switching between QPSK and QAM-16 modulation while bandwidth and transmit power are fixed. The proposed system is implemented with a \SI{-40}{dBc} bandwidth of \SI{\measBW}{\kilo\hertz} and a transmit power of \SI{\measPWR}{dBm}.

The adaptive quality feature is verified and the system changes quality immediately when the detected error rate drops below a predefined threshold. The measured bit error rates are \measBERQAMGood  and \measBERQPSKBad for high and low data rates respectively. The total delay from transmitter to receiver is measured to \SI{\measDelay}{\milli\second}, making the system well suited the for two-way communication as well as broadcasting. 
\end{abstract}

% INTRODUCTION
\section{Introduction}
\label{sec:Introduction}
This is the introduction



% SPECIFICATIONS
% !TEX root = main.tex
\section{System Specifications}
\label{sec:specifications}
The proposed radio communication system switches between QPSK and QAM-64 modulation, at a fixed transmit power and bandwidth, in order to obtain adaptable sound quality.  The system use the 2.4GHz ISM band with a carrier frequency of 2.415GHz. The system is designed for a transmission distance of 10 meter in an indoor environment with a bandwidth of \bw kHz. Some key specifications of the system is listed in table \ref{tab:specs_data} and \ref{tab:specs_ber}. Table \ref{tab:specs_data} shows the parameters for the data path and the values are listed for low / high data rate transmission. Table \ref{tab:specs_ber} shows parameters for the simpler BER path. 
% !TEX root = main.tex
% Table generated by Excel2LaTeX from sheet 'specs'
\begin{table}[htbp]
  \centering
  \caption{System specifications - data path}
    \begin{tabular}{lc}
    \rowcolor[rgb]{ 0,  0,  0} \textcolor[rgb]{ 1,  1,  1}{\textbf{System Variables}}	& \textcolor[rgb]{ 1,  1,  1}{\textbf{Value}} 		\\
    \rowcolor[rgb]{ 0,  0,  0} \textcolor[rgb]{ 1,  1,  1}{} & \textcolor[rgb]{ 1,  1,  1}{\textbf{Low / High Data rate}} 				\\
    	Frequency $f_0$ 								& $\MainCarrierFreq$ MHz 							\\
    	Modulation 									& $\text{QPSK} / \text{QAM-64}$						\\
    	Bit per symbol $m$ 								& $2 /6$ 											\\
    	Sound sampling rate $f_s$  						& $11025 / 22050$ Hz 								\\
    	Bits per sound sample $b_s$ 						& $8 / 12$ bits 										\\
    	Sound datarate $R_{ss}$ 							& $\sourceDataRateQPSK / \sourceDataRateQAM$ kbits/s	\\
    	Channel coding 								& Hamming (4,7) 									\\

    \rowcolor[rgb]{ 0,  0,  0} \textcolor[rgb]{ 1,  1,  1}{\textbf{Packet Parameters}} & \textcolor[rgb]{ 1,  1,  1}{} 				\\
	Packet header size      							& $\headerBits $  bits								\\
    	Packet data length     							& $\packetDataSymbols$  symbols					 	\\
    	Packet size   									& $\packetDataBitsQPSK/ \packetDataBitsQAM$  bits		\\
    
    \rowcolor[rgb]{ 0,  0,  0} \textcolor[rgb]{ 1,  1,  1}{\textbf{Frame Parameters}} & \textcolor[rgb]{ 1,  1,  1}{} 				\\
    	Training sequence type 							& Barker										 	\\
    	Training sequence length							& $\barkerSymbols$ symbols 					 		\\
   	Training sequence size bits 						& $\barkerBitsQPSK / \barkerBitsQAM$ bits	 			\\
    	Frame size 									& $\frameSizeBitsQPSK / \frameSizeBitsQAM$ bits			\\
        
    \rowcolor[rgb]{ 0,  0,  0} \textcolor[rgb]{ 1,  1,  1}{\textbf{Burst Parameters}} & \textcolor[rgb]{ 1,  1,  1}{} 					\\
    	Guard period 									& $\guardSymbols$ symbols							\\
    	Burst size 										& $\burstSizeBitsQPSK / \burstSizeBitsQAM$ bits 			\\
    	
    \rowcolor[rgb]{ 0,  0,  0} \textcolor[rgb]{ 1,  1,  1}{\textbf{Transmission Characteristics}} & \textcolor[rgb]{ 1,  1,  1}{} 		\\
    	System bit rate $R_b$ 							& $191,35 / 577,18$ kbits/s 							\\
    	Symbol rate $R_s$ 								& $95,67 / 96,20$ ksymbols/s 							\\
    	Pulse shaping filter 								& root raised cosine 									\\
    	Pulse shaping filter parameter $\alpha$ 				& $0.3$ 											\\
    	Minimum signal bandwidth $\Delta f$ 				& $62,2 / 62,5$ kHz 									\\
    \end{tabular}
  \label{tab:specs_data}
\end{table}

\begin{table}[htbp]
  \centering
  \caption{System specifications - ber path}
    \begin{tabular}{lc}
    \rowcolor[rgb]{ 0,  0,  0} \textcolor[rgb]{ 1,  1,  1}{\textbf{System Variables}}	& \textcolor[rgb]{ 1,  1,  1}{\textbf{Value}} 		\\
    \rowcolor[rgb]{ 0,  0,  0} \textcolor[rgb]{ 1,  1,  1}{} & \textcolor[rgb]{ 1,  1,  1}{\textbf{Low / High Data rate}} 				\\
    	Frequency $f_0$ 								& $\BERCarrierFreq$ MHz 							\\
    	Modulation 									& $\text{QPSK}$									\\
	
    \rowcolor[rgb]{ 0,  0,  0} \textcolor[rgb]{ 1,  1,  1}{\textbf{Packet Parameters}} & \textcolor[rgb]{ 1,  1,  1}{} 				\\
	Packet header size      							& $\headerBits $  bits								\\
    	Packet size									& $\BERDataBits$  bits					 			\\
    
    \rowcolor[rgb]{ 0,  0,  0} \textcolor[rgb]{ 1,  1,  1}{\textbf{Frame Parameters}} & \textcolor[rgb]{ 1,  1,  1}{} 				\\
    	Training sequence type 							& Barker										 	\\
    	Training sequence length							& $\barkerSymbols$ symbols 					 		\\
   	Training sequence size bits 						& $\barkerBitsQPSK$ bits	 							\\
    	Frame size 									& $\BERFrameSize$ bits								\\
    	
    \rowcolor[rgb]{ 0,  0,  0} \textcolor[rgb]{ 1,  1,  1}{\textbf{Transmission Characteristics}} & \textcolor[rgb]{ 1,  1,  1}{} 		\\
    	System bit rate $R_b$ 							& $\BERDataRate$ kbits/s 							\\
    	Symbol rate $R_s$ 								& $\BERSymbolRate$ ksymbols/s 						\\
    	Pulse shaping filter 								& root raised cosine 									\\
    	Pulse shaping filter parameter $\alpha$ 				& 0.3 											\\
    	Minimum signal bandwidth $\Delta f$ 				& $62,2 / 62,5$ kHz 									\\
    \end{tabular}
  \label{tab:specs_ber}
\end{table}


The burst format for the transmitted data is shown in figure \ref{fig:burst_format}. The bursts are different when using QPSK and QAM-64 modulation, because the same number of e.g. training symbols maps to a different number of bits. The data packages (a and b) is transmitted continuously with the indicated guard period. The BER packages is very small compared to the data packages, and is only transmitted ones per received data package. Thus no guard period is specified for these packages. 
% !TEX root = main.tex
\begin{figure} 
    \centering
  \subfloat[QPSK data package\label{1a}]{%
\begin{tikzpicture}[                
                    slot/.style={
            		text centered,
			font=\scriptsize,
			align=center,
			anchor=center,
			minimum height=0.7cm
            	}]
\draw[thick] (0,0) rectangle (\linewidth, -0.7);
\draw[thick] (0.25\linewidth,0) -- ++(0, -0.7);
\draw[thick] (0.35\linewidth,0) -- ++(0, -0.7);
\draw[thick] (0.85\linewidth,0) -- ++(0, -0.7);

\draw
(0,0) node[slot, minimum width=0.25\linewidth, anchor=north west](barker){Training sequence \\ \barkerBitsQPSK}
(0.25\linewidth,0)  node[slot, minimum width=0.1\linewidth, anchor=north west](barker){Header \\ \headerBits}
(0.35\linewidth,0) node[slot, minimum width=0.43\linewidth, anchor=north west](barker){Speech data \\ \packetDataBitsQPSK}
(0.85\linewidth,0) node[slot, minimum width=0.15\linewidth, anchor=north west](barker){Guard bits \\ \guardBitsQPSK}
;	

\end{tikzpicture}
}
  \\
  \subfloat[QAM-64 data package\label{1b}]{%
\begin{tikzpicture}[                
                    slot/.style={
            		text centered,
			font=\scriptsize,
			align=center,
			anchor=center,
			minimum height=0.7cm
            	}]
\draw[thick] (0,0) rectangle (\linewidth, -0.7);
\draw[thick] (0.25\linewidth,0) -- ++(0, -0.7);
\draw[thick] (0.35\linewidth,0) -- ++(0, -0.7);
\draw[thick] (0.85\linewidth,0) -- ++(0, -0.7);

\draw
(0,0) node[slot, minimum width=0.25\linewidth, anchor=north west](barker){Training sequence \\ \barkerBitsQAM}
(0.25\linewidth,0)  node[slot, minimum width=0.1\linewidth, anchor=north west](barker){Header \\ \headerBits}
(0.35\linewidth,0) node[slot, minimum width=0.43\linewidth, anchor=north west](barker){Speech data \\ \packetDataBitsQAM}
(0.85\linewidth,0) node[slot, minimum width=0.15\linewidth, anchor=north west](barker){Guard bits \\ \guardBitsQAM}
;	

\end{tikzpicture}}
\\
  \subfloat[QPSK BER package\label{1c}]{%
\begin{tikzpicture}[                
                    slot/.style={
            		text centered,
			font=\scriptsize,
			align=center,
			anchor=center,
			minimum height=0.7cm
            	}]
\draw[thick] (0,0) rectangle (\linewidth, -0.7);
\draw[thick] (0.70\linewidth,0) -- ++(0, -0.7);
\draw[thick] (0.82\linewidth,0) -- ++(0, -0.7);

\draw
(0,0) node[slot, minimum width=0.7\linewidth, anchor=north west](barker){Training sequence \\ \barkerBitsQPSK}
(0.70\linewidth,0)  node[slot, minimum width=0.12\linewidth, anchor=north west](barker){Header \\ \headerBits}
(0.82\linewidth,0) node[slot, minimum width=0.18\linewidth, anchor=north west](barker){BER data \\ \BERDataBits}
;	

\end{tikzpicture}}
  \caption{Burst format for QPSK(a) and QAM-64(b) modulated data bursts, and QPSK modulated BER burst (c)}
  \label{fig:burst_format} 
\end{figure}
 

% LINK BUDGET 
% !TEX root = main.tex
\section{Link Budget}
\label{sec:link_budget}
The link budget for the system is shown in table \ref{tab:link_budget}. As the purpose of the system is to demonstrate the feedback feature, the system is designed for the test environment only, which is reflected in the link budget. The system is designed to operate indoors with a distance of 10 meters between transmitter and receiver. Table \ref{tab:link_budget} shows losses from path loss, loss in TX and RX and some estimated key parameters at the receiver, such as $E_b/N_0$ and BER. 

% !TEX root = main.tex
\begin{table}[htbp]
  \centering
  \caption{Link Budget}
    \begin{tabular}{lccccr}
    \rowcolor[rgb]{ 0,  0,  0} \textcolor[rgb]{ 1,  1,  1}{\textbf{TX Loss}}	& \textcolor[rgb]{ 1,  1,  1}{\textbf{Value}} 		\\
    \rowcolor[rgb]{ 0,  0,  0} \textcolor[rgb]{ 1,  1,  1}{} & \textcolor[rgb]{ 1,  1,  1}{\textbf{Low / High Data rate}} 		\\
    PA Power, $P_{PA}$ 						& $10 \text{dBm}$											\\
    TX Connector Loss, $L_{ConT}$  				& $-0.3 \text{dB}$ 											\\
    TX Power, $P_T$ 							& $9.4 \text{dBm}$											\\
    TX Antenna Gain, $G_T$ 					& $3 \text{dBi}$ 											\\
    Effective (Isotropic) Radiated Power, EIRP  		& $12.4 \text{dBm}$										\\
    
    \rowcolor[rgb]{ 0,  0,  0} \textcolor[rgb]{ 1,  1,  1}{\textbf{Path Loss}}
    & \textcolor[rgb]{ 1,  1,  1}{\textbf{}} 															\\
    Distance, $d$  							& $2 \text{m}$ 											\\
    Floor loss factor, $Pf(n)$ 					& $0 \text{dB}$											\\
    Distance power loss coefficient, $N$ 			& $38$ 												\\
    Total ITU path loss, $L_P$ 					& $-51.1 \text{dB}$										\\
    
    \rowcolor[rgb]{ 0,  0,  0} \textcolor[rgb]{ 1,  1,  1}{\textbf{RX Loss}}	& \textcolor[rgb]{ 1,  1,  1}{\textbf{}} 			\\
    RX antenna gain, $G_R$					& $3 \text{dBi}$ 											\\
    RX connector loss, $L_{ConR}$ 				& $-0.3 \text{dB}$ 											\\
    Total RX Loss, $L_R$						& $2.4 \text{dB}$											\\
    Total Received Power, $P_R$ 				& $-36.3 \text{dBm}$										\\
    
    \rowcolor[rgb]{ 0,  0,  0} \textcolor[rgb]{ 1,  1,  1}{\textbf{RX Loss}}	& \textcolor[rgb]{ 1,  1,  1}{\textbf{}} 			\\
    Antenna Noise Density, $N_0$ 				& $-145.73 \text{dbm/Hz}$								\\
    Antenna Total Noise Power, $N$   				& $-97.806 \text{dBm}$										\\
    RX Noise Figure, $NF$ 					& $7.0 \text{dB}$									\\
        
    \rowcolor[rgb]{ 0,  0,  0} \textcolor[rgb]{ 1,  1,  1}{\textbf{RX Properties}}	& \textcolor[rgb]{ 1,  1,  1}{\textbf{}} 		\\
    Carrier-to-noise ratio, $C/N$ 					& $18.548 \text{dB}$										\\
    Eb over N0, $E_b/N_0$ 					& $13.667 \text{dB} / 8.893 \text{dB}$					\\
    \end{tabular}
  \label{tab:link_budget}

\end{table}


As the system is designed to switch between two different modulation formats the received \ebnot should not be carefully tuned. The link budget is designed such that the \ebnot is mostly good enough for QAM-64 modulation under line of sight (LOS), but forces the system to switch to QPSK if the LOS is lost. 

The different parts of the link budget will be discussed in this section.

\subsection{TX and RX Loss}
\label{sec:txandrxloss}
The value for connector loss is taken from datasheets of standard coaxial RF connectors \cite{rfconnector}. The antenna gain value is taken from the data sheet \cite{antenna} which reports a peak gain of 3.4 dBi. We used the value 3 dBi in the link budget to account for suboptimal conditions. The launch power, PA Power, was adjusted after measurements to obtain appropriate $E_b/N_0$ at the receiver.

\subsection{Path Loss}
\label{sec:path_loss}
The estimated path loss constitutes solely of the propagation loss obtained from the ITU Indoor Propagations Loss Model \cite{itu_model}. The loss model consists of two adjustable factors, the distance power loss coefficient, $N$, and the floor loss penetration factor, $P_f(n)$. The latter is set to 0, and the former was set to 38 after calibrating the test environment. Other loss factors such as pointing loss and polarisation loss was considered but measurements showed that the amount of reflections in the room made pointing and polarisation irrelevant to the received power. More details on the measurements is given in section \ref{sec:verification}. 

  
\subsection{RX Noise}
\label{sec:rxnoise}
The antenna noise density was measured with a spectrum analyser and the estimated value was taken as an average of several single runs. The noise figure of the receiver is included to account for noise added by the radio hardware, with value taken from the data sheet. The small scale fading margin, $M_{ssf}$, is included to account for variations in received power. This margin was obtained by evaluating several measurements of received power using a spectrum analyser. The particular value is taken to be two times the standard deviation of the measured values. 

\subsection{RX Properties}
\label{sec:rxproperties}
Some key properties of the received signal is calculated based on the estimated values in the link budget. The BER and $E_b/N_0$ is calculated for the two modulation schemes separately. The bit error rate is calculated for QPSK and QAM-64 by equation \ref{eq:berqpsk} and \ref{eq:berqam} respectively.

\begin{equation}
\label{eq:berqpsk}
P_B \approx \frac{1}{2}\erfc\sqrt{\frac{E_b}{N_0}}
\end{equation}

\begin{equation}
\label{eq:berqam}
P_B \approx \frac{2}{\log_2M}\left(1-\frac{1}{M}\right)\erfc\left(\sqrt{\frac{3\log_2M}{2(M-1)}\cdot \frac{E_b}{N_0}}\right)
\end{equation}




\newpage
% DESIGN DESCRIPTION
% !TEX root = main.tex
\section{Design Description}
\label{sec:design_description}
Figure \ref{fig:block_toplevel} shows how the transmitter and receiver communicates within the system. Block diagrams for the two subsystems are shown in appendix \ref{a:block_diagram}, figure \ref{fig:block_diagram} and \ref{fig:block_diagram_feedback}. The data rate between each block of the data path is indicated with the thin arrows. The behaviour of the system will be explained in this section. 

Because two different modulation schemes are used, the receiver need to know how to decode the incoming data packets. This problem is solved by using two different training sequences in the beginning of each frame. As shown in figure \ref{fig:block_diagram}, after frame sync, the receiver perform a check on the received Barker sequence before de-mapping the symbols. 

In the transmitter, a variable called \textit{Session State} keep the information about what data quality and modulation scheme to use. As figure \ref{fig:block_diagram} indicates, the session state influences several blocks of the TX side of the transmitter. The decision of when to change data quality is left completely to the transmitter. For every received data packet, the receiver computes the number of detected errors and transmit this number back to the transmitter. These BER packets are always transmitted using QPSK-modulation. Based on the received number of detected errors the transmitter decides whether to change session state or not. 

The sound producer and sound consumer contains functionality for handling the sound input and output to the sound card of the computer. They are implemented using the Windows API \cite{WinAPI}. Sound producer reads sound samples from the sound card at full quality (16 bit, 44100 Hz stereo) and writes the samples to a queue accessible for the source encoder. Sound consumer equivalently reads sound samples from a queue controlled by the unpacking block and writes to the computer sound card. 

The source encoder performs lossy compression of the produced sound samples. The bit resolution is reduced to 12 bit per sound sample, and the sampling rate is reduced by a factor 2 or 4 depending on the session state. The source decoder performs the inverse operation, writing 2 or 4 copies of the same sample to the sound consumer.  

In the packing block, sound data is read from the source encoder, a header is added, and the packet is sent to the packet queue. The eight bit packet header consist of a three bit session ID and a five bit packet ID. 

A scrambler is implemented before FEC, which computes a bitwise XOR between a pseudo random bit string and the packet. The bit string used for scrambling is of the same size as the packet itself. The scrambler performs the exact same operation at RX and TX.

The implemented FEC algorithm is Hamming (7,4). The implementation is a fast, pre-written C-code, written by Michael Dipperstein \cite{hamming}. 

The system uses Grey Code for mapping the binary data to a complex vector $z$. The mapping schemes are shown in figure \ref{fig:mapping}.

% !TEX root = main.tex
\begin{figure} 
    \centering
  \subfloat[\label{1a}]{%
  \includegraphics[width=0.3\linewidth]{qpsk_mapping.pdf}
  
}
\\ 
  \subfloat[\label{1b}]{%
    \includegraphics[width=0.6\linewidth]{qam16mapping.pdf}
}

  \caption{Symbol mapping for QPSK (a) and QAM-16 (b) modulated symbols}
  \label{fig:mapping} 
\end{figure}


The training sequence is added to both I and Q symbols in the block \textit{Add Barker}. The training sequence is an appropriate repetition of Barker sequences of length 7 and 13 for QPSK and QAM-16 modulation respectively. The total length of the training sequence is 26 symbols. 

In the last step before transmission, the symbols are upsampled by a factor $\sps$ and filtered with a pulse shaping filter. The filter is Root Raised Cosine with a roll-off factor of 0.3. The same filter is applied as a matched filter in the first step at the receive side before the samples are downsampled again. 

The USRP is configured to transmit continuously with a symbol rate of $\symbolRateQPSK$ symbols per second and transmit arrays of zeros when no data is available. The USRP interface is implemented using the NI-USRP DLL \cite{labviewDLL}.

Symbol synchronisation is done by choosing the sample offset that maximizes the signal energy. 

Frame synchronisation is done by computing the crosscorrelation between the two training sequences and the received symbols. The value of the crosscorrelation is compared to a pre set threshold. The first index that gives a value exceeding this threshold is considered to be the beginning of the frame. After a frame is found, only the frame symbols are passed along to the next block.

\subsection{Frequency and Phase Synchronisation}
The implemented frequency synchronisation algorithm is based on the Mth-power algorithm\cite{viterbi}, which is further improved by using a Kalman filter. Frequency offset is estimated from the training sequence only. The Mth-power algorithm estimates the phase offset of MPSK modulated symbols, by mapping all symbols to the same point in the complex plane. By tracking the angular difference between consecutive symbols, the frequency offset may be estimated. In our case, the QPSK symbols of the training sequence is raised to the 4th power, and the phase is estimated as indicated in figure \ref{fig:mth_pow}.

% !TEX root = main.tex
\begin{figure}[htbp]
\centering
\begin{tikzpicture}
% Grids
\coordinate (origo) at (0,0);
\coordinate (origo_w) at (5.1,0);

\draw[step=0.3cm,gray,very thin] (-1.51,-1.51) grid (1.5,1.5);
\draw[step=0.3cm,gray,very thin] (3.6,-1.51) grid (6.6, 1.5);

% Axis
\draw[thick, ->] (-1.5,0) -- (1.5, 0) node [anchor=north west]{\Re[$z$]};
\draw[thick, ->] (0,-1.5) -- (0,1.5) node [anchor=south west]{\Im[$z$]};
\draw[thick, ->] (3.6,0) -- (6.6, 0) node [anchor=north west]{\Re[$w$]};
\draw[thick, ->] (5.1,-1.5) -- (5.1,1.5) node [anchor=south west]{\Im[$w$]};

\draw[thick] (0.9, 0) ++(0, 2pt) -- ++(0,-4pt) node[font=\scriptsize, yshift=-0.1cm, xshift=-0.1cm]{1};
\draw[thick] (-0.9, 0) ++(0, 2pt) -- ++(0,-4pt) node[font=\scriptsize, yshift=-0.1cm, xshift=-0.1cm]{-1};
\draw[thick] (0, 0.9) ++(2pt, 0) -- ++(-4pt, 0) node[font=\scriptsize, yshift=-0.1cm, xshift=-0.1cm]{1};
\draw[thick] (0, -0.9) ++(2pt, 0) -- ++(-4pt, 0) node[font=\scriptsize, yshift=-0.1cm, xshift=-0.1cm]{-1};

\draw[thick] (6, 0) ++(0, 2pt) -- ++(0,-4pt) node[font=\scriptsize, yshift=-0.1cm, xshift=-0.1cm]{4};
\draw[thick] (4.2, 0) ++(0, 2pt) -- ++(0,-4pt) node[font=\scriptsize, yshift=-0.1cm, xshift=-0.1cm]{-4};
\draw[thick] (5.1, 0.9) ++(2pt, 0) -- ++(-4pt, 0) node[font=\scriptsize, yshift=-0.1cm, xshift=-0.1cm]{4};
\draw[thick] (5.1, -0.9) ++(2pt, 0) -- ++(-4pt, 0) node[font=\scriptsize, yshift=-0.1cm, xshift=-0.1cm]{-4};

% Plot
\draw[blue,fill=blue] (0.9,0.9) circle (2pt) coordinate[](sample);
\draw[blue,fill=blue] (-0.9,0.9) circle (2pt);
\draw[blue,fill=blue] (0.9,-0.9) circle (2pt);
\draw[blue,fill=blue] (-0.9,-0.9) circle (2pt);
\draw[red,fill=red] (0.73,1.043) circle (2pt) coordinate[](sample_err);

\draw[thin, blue] (0,0) -- (sample);
\draw[thin, red] (0,0) -- (sample_err);
\pic [draw, red, ->, angle eccentricity=1.5] {angle = sample--origo--sample_err};
\draw (0.3, 0.3) node[anchor=north west, text=red, yshift=0.2cm]{$\phi$};

\draw[blue,fill=blue] (3.83,0) circle (2pt) coordinate[](sample_w);
\draw[red,fill=red] (4.125,-0.818) circle (2pt) coordinate[](sample_err_w);
\draw[thin, blue] (origo_w) -- (sample_w);
\draw[thin, red] (origo_w) -- (sample_err_w);
\pic [draw, red, ->, angle eccentricity=1.5] {angle = sample_w--origo_w--sample_err_w};
\draw (origo_w) ++(-1.1,-0.35) node[anchor=north west, text=red, yshift=0.2cm]{$4\phi$};

% Other
\draw[thick, ->] (2, 1) .. controls (2.5,1.1) .. (3,1) node[above, anchor=south east, xshift=0.1cm]{$w=z^4$} ;

\end{tikzpicture}
\caption{Illustration the mapping $w=z^4$ which indicates how a phase error in QPSK modulated signal can be observed.}
\label{fig:mth_pow}
\end{figure}

For each received frame, the frequency offset is estimated by the mean of the angular difference between all consecutive symbols of the training sequence. The estimate obtained from a single frame at time $k$  is referred to as $\widetilde{\omega}_k$.  

The accuracy of the estimate is improved by using a Kalman filter. The state of the Kalman filter is the one-dimensional state vector $\omega_k$ which is the true frequency offset between the transmitter and the receiver at time $k$ (i.e. frame $k$). The state is represented by the \textit{a posteriori} state and variance estimate $\widehat{\omega}_{k | k}$ and $\widehat{\sigma}_{k | k}$, where the subscript $n | m$ indicates the estimate of time $n$ given observations up to and including time $m \leq n$. 

The estimate is obtained in two stages, referred to as the \textit{prediction stage} and the \textit{update stage}. In the prediction stage, the \textit{a priori} estimates $\widehat{\omega}_{k | k-1}$ and $\widehat{\sigma}_{k | k-1}$ is obtained by equations \ref{eq:kalman_pred_wk} and \ref{eq:kalman_pred_sigmak}. $\sigma_f$ is the stationary variance of the process noise and must be tuned carefully in order to obtain appropriate convergence speed. In our system $\sigma_f$ was set to $0.8\cdot 10^{-6}$.  

\begin{gather}
\widehat{\omega}_{k | k-1} = \widehat{\omega}_{k-1 | k-1}  \label{eq:kalman_pred_wk} \\
\widehat{\sigma}_{k | k-1} = \widehat{\sigma}_{k-1 | k-1} + \sigma_f \label{eq:kalman_pred_sigmak} 
\end{gather} 

In the update stage, the \textit{a posteriori} estimates $\widehat{\omega}_{k | k}$ and $\widehat{\sigma}_{k | k}$ is obtained by equations \ref{eq:kalman_up_wk} and \ref{eq:kalman_up_sigmak}. $\widetilde{\omega}_k$ and $\widetilde{\sigma}_{k}$ is the observed frequency offset and estimated observation noise at time $k$ respectively.

\begin{gather}
\widehat{\omega}_{k | k} = \widehat{\omega}_{k | k-1} + \frac{\widehat{\sigma}_{k | k-1}}{\widehat{\sigma}_{k | k-1} + \widetilde{\sigma}_{k}} ( \widetilde{\omega}_{k} - \widehat{\omega}_{k | k-1} )  \label{eq:kalman_up_wk} \\
\widehat{\sigma}_{k | k} = \widehat{\sigma}_{k | k-1} \frac{\widetilde{\sigma}_{k}}{\widetilde{\sigma}_{k} + \widehat{\sigma}_{k | k-1}  } \label{eq:kalman_up_sigmak} 
\end{gather} 

After the estimates $\widehat{\omega}_{k | k}$ and $\widehat{\sigma}_{k | k}$ is obtained, $\widehat{\omega}_{k | k}$ is applied to each symbol in frame $k$. The phase offset is then estimated as the mean of the angular deviation between the training sequence of the received symbols, and the ideal Barker sequence.




% DESIGN MOTIVATION
% !TEX root = main.tex
\section{Design Motivation}
\label{sec:design_motivation}
The goal of the proposed system is to demonstrate a radio communication system with adaptable sound quality. This main goal is the background for all design choices that is made. In this section, the motivation behind the key specifications and some of the solutions described in section \ref{sec:design_description} is given.

\subsection{System specifications}
\subsubsection{Transmit Power}
As shown in table \ref{tab:link_budget}, the transmitter PA power, $P_{PA}$, is set to \paPower dBm. As the purpose of the system is to demonstrate the adaptable quality feature, the transmit power was tuned to a value suitable for this purpose. Too high transmit power would give a received $E_b/N_0$ high enough to always transmit at the best quality, thus disabling us from demonstrating the quality adoption. 

\subsubsection{Modulation Scheme}
The system uses QPSK for low data rate transmission and QAM-16 for high data rate transmission. QPSK has the advantage of constant symbol amplitude and therefore reduces nonlinear effects in the amplifiers. Because of the orthogonality between the I and Q signals, QPSK also gives twice the capacity as BPSK for same $E_b/N_0$ sensitivity. QAM-16 is chosen for high data rates because the information content in each symbol is twice as big as for QPSK, making the difference big enough to demonstrate an audible effect.

\subsubsection{Bandwidth}
The motivation behind the adaptable quality feature, is to maximise data rate, for fixed bandwidth and transmit power. To demonstrate this feature, the system is designed width a fixed bandwidth, equal for both quality levels. The specific value for the bandwidth is however not important for this demonstration. The transmitted symbol rate is chosen large enough to enable both QPSK and QAM-16 transmission. The ideal Nyquist bandwidth is calculated from the symbol rate and the properties of the pulse shaping filter, and presented in table \ref{tab:link_budget}. 

%\subsubsection{Training Sequence}
%Barker codes is chosen as training sequence because it is the unique sequence with ideal autocorrelation property \cite{barker}. A total length of 26 training symbols is chosen to make the

 
\subsection{System Architecture}
The adaptable quality feature requires some additional logic, compared to a system without a feedback path. One important design choice is to determine where the decision of changing quality level (referred to as \textit{state}) is to be made. During the design of the proposed system, two solutions were considered:
\begin{enumerate}
\item 
The receiver decides when to change state and sends a simple message to the transmitter, telling it to change state. The receiver has its own state variable, and updates it after sending the message to the transmitter.

The advantage of this solution is that the receiver always know how the received symbols are modulated. Thus, no state information is needed in the transmitted packet. 

The draw back is that there will be some delay from the receiver ask for a change of state, until state change is completed. This will either cause a few packets to get lost in the meantime, or extra logic must be implemented to handle the issue. Extra logic must also be implemented to handle the case of bit errors in the ``change-state-message''.

\item
The transmitter decides when to change state and the receiver continuously send information about detected error rate to the transmitter. This requires information about the state to be contained in the packet and the receiver will need more complexity in order to determine the incoming modulation format. 
However, this will eliminate the senario when the transmitter and the receiver are in different states. 
\end{enumerate}
We chose to implement the second solution. The state information is kept in the two different Barker sequences. Because we would have to search for one Barker sequence in the frame synchronisation anyway, searching for one more would be both computationally fast and easy to implement. In addition, the good autocorrelation properties of the Barker sequence makes the state information robust against noise, and no extra FEC is needed to keep this information safe. 

\subsection{Implemented Functionality}
\subsubsection{Source Enconding}
We chose not to implement any source encoding except for the primitive reduction of data rate. A source encoder removes redundancy in the speech data and increase the information content of every transmitted symbol. This functionality is not necessary to demonstrate the adaptable quality and is therefore not implemented. 

The source encoder also makes the system more sensitive to bit errors, as the information content in each bit is increased. Our system is designed to operate at $E_b / N_0$ low enough to demonstrate the quality adaption. When the received $E_b / N_0$ falls low enough to make the system change from QAM-16 to QPSK modulation, we would like the system to have a large margin before the BER for QPSK also get too high and the link is broken. Using a source encoder would make this window smaller and the system would be harder to demonstrate. 

\subsubsection{Scrambler}
Analysis of the received constellation diagram clearly showed that some symbols occurred more frequently at the receiver than others. Both the RF hardware and the pulse align algorithm are highly sensitive to repetitive symbol sequences. When comparing the received constellation from speech data to the constellation from random data, we could see that the quality was considerably increased when transmitting white data. We therefore implemented a scrambler to whiten the speech data before transmission. 

\subsubsection{FEC}
The implemented FEC algorithm is Hamming(7,4) and serves two purposes. In the data path, the receiver must estimate the number of bit errors and send this number back to the transmitter. In the BER path, the transmitter use the FEC to determine if the received BER packet is valid or not. If the BER packet contains errors, the packet is considered as invalid in order to reduce the risk of erroneous changes of quality level. 

This particular algorithm was chosen for its simplicity and decent trade-off between redundancy overhead and minimal Hamming distance. The algorithm allows for detection of any two-bit errors in every 7 bit code word, which was considered sufficient for this purpose. In addition, pre-written C-code implementation exists \cite{hamming} which saved development time. More advanced FEC algorithms should be considered for applications where bandwidth limitations require less overhead, or error correction is necessary. 

\subsubsection{Frequency and Phase Synchronisation}
The Mth-power algorithm is chosen for frequency synchronisation because of the good accuracy and simple implementation. By estimating the frequency offset as the mean of the angular deviation between all consecutive symbols, the ML estimate of the frequency offset is obtained. In addition, the Mth-power algorithm does not require a known training sequence, and may be applied to all data samples when QPSK modulation is being used. By extending the algorithm to a Kalman filter, the mean square error between the estimated offset and the actual offset is minimised. The improvement from the Kalman filter is clearly seen in figure \ref{fig:kalman_freq} which shows the estimated frequency offset at the receiver with and without Kalman filter.

\begin{figure}[htbp]
\centering
\includegraphics[width=\figW\linewidth]{KalmanFreq.png}
\caption{Estimated frequency offset with and without Kalman filter}
\label{fig:kalman_freq}

\end{figure}

The Mth-power algorithm is however not linear, making it less suited for phase estimation. Therefore, the linear algorithm based on comparing the phase directly to the training sequence is used for estimating the phase.
 







% VERIFICATION
% !TEX root = main.tex
\section{Measurements and Verification}
\label{sec:verification}
In this section we present measurements and verifications of the proposed system. We first present a verification of the system specifications in section \ref{sec:specs_verification}. In section \ref{sec:perf_meas} the system performance will be evaluated under different conditions, and some key performance parameters will be presented.

\subsection{Verification of System Parameters}
\label{sec:specs_verification}
The verified system specifications is summarised in table \ref{tab:meas_specs}.
% !TEX root = main.tex
\begin{table}[htbp]
  \centering
  \caption{Summarised Measurements of System Parameters}
    \begin{tabular}{lc}
    \rowcolor[rgb]{ 0,  0,  0} \textcolor[rgb]{ 1,  1,  1}{\textbf{System Parameter}}	& \textcolor[rgb]{ 1,  1,  1}{\textbf{Measured Value}} 		\\
    	Half-Power Bandwidth										& \SI{\measBW}{kHz}					\\
    	PA power, $P_{PA}$ 											& \SI{\measPWR}{dBm}					\\
    	Delay 													& \SI{\measDelay}{ms} 					\\
 \end{tabular}
  \label{tab:meas_specs}
\end{table}

The power spectrum of the signal was analyzed in MATLAB's signal analyzer toolbox \cite{signalAnalyzer}. The baseband power spectrum is shown in figure \ref{fig:pwr_spectrum} together with the theoretical spectrum. The -40dBc bandwidth is measured to $\measBW \si{kHz}$.
The transmit power was verified by first calibrating the USRP power spectral density using a spectrum analyser, and then calculating the total transmit power from measured bandwidth\footnote{Due to the CIVID-19 situation the lab equipment was not available for measurements on the implemented system}. This way, the transmit power was measured to $\measPWR \si{dBm}$.

\begin{figure}[htbp]
\begin{center}
\includegraphics[width=\figW\linewidth]{spectrum.png}
\caption{Received signal power spectrum}
\label{fig:pwr_spectrum}
\end{center}
\end{figure}

The system delay was measured by transmitting a known bit sequence and evaluating the time delay from sound producer to sound consumer. The delay was measured by running both the transmitter and receiver software on the same computer. The average measured delay was $\measDelay \si{ms}$ with a sample standard deviation of $\measDelayStd \si{ms}$.  

\subsection{Performance Measurements}
\label{sec:perf_meas}
The system is tested in an indoor environment with both USRP's at the same height, 1 meter apart and no polarization mismatch\footnote{The measurements had to be performed in a tiny student apartment because the lab was closed}. For the sake of reproducibility, we chose to simulate the varying state of the radio channel by adjusting the transmit power instead of changing the radio channel physically. Two different transmit power levels was used to verify transmission at both qualities. High quality transmission was measured with a transmit power of $\measPWR \si{dBm}$ and low quality with $\measPWRbad \si{dBm}$. 

Eye diagram and constellation diagram is provided for both transmit power levels and both modulation formats, together with some key performance measurements. The diagrams for high and low transmit power are summarised in figure \ref{fig:good_diagrams} and \ref{fig:bad_diagrams} respectively. Estimated performance parameters are summarised in table \ref{tab:meas_params_good} and \ref{tab:meas_params_bad} for high and low transmit power respectively. In the tables, BER is the true measured bit error rate and error vector magnitude (EVM) and SNR is estimated from the constellation diagrams. EVM is the average magnitude of the error vector normalized to the peak constellation power. \ebnot is estimated using the measured -40dBc bandwidth for noise power.

% !TEX root = main.tex
\begin{figure} 
    \centering
  \subfloat[QPSK constellation\label{1a}]{%
\includegraphics[width=0.45\linewidth]{constellationQPSKgood.png}
}
  \subfloat[QAM-16 constellation\label{1b}]{%
\includegraphics[width=0.45\linewidth]{constellationQAMgood.png}

}
\\
  \subfloat[QPSK eye diagram\label{1c}]{%
\includegraphics[width=0.45\linewidth]{eyeQPSKgood.png}

}
  \subfloat[QAM-16 eye diagram\label{1d}]{%
\includegraphics[width=0.45\linewidth]{eyeQAMgood.png}

}
  \caption{Eye diagram and constellation for QPSK (a and c) and QAM-16 (b and d) modulated symbols. Transmit power: $\measPWR \si{dBm}$}
  \label{fig:good_diagrams} 
\end{figure}

% !TEX root = main.tex
\begin{figure} 
    \centering
  \subfloat[QPSK data package\label{1a}]{%
\includegraphics[width=0.45\linewidth]{constellationQPSKgood.png}
}
  \subfloat[QAM-64 data package\label{1b}]{%
\includegraphics[width=0.45\linewidth]{constellationQPSKgood.png}

}
\\
  \subfloat[QPSK BER package\label{1c}]{%
\includegraphics[width=0.45\linewidth]{eyeQPSKgood.png}

}
  \subfloat[QAM-64 data package\label{1d}]{%
\includegraphics[width=0.45\linewidth]{eyeQPSKgood.png}

}
  \caption{Eye diagram and constellation for QPSK (a and c) and QAM-16 (b and d) modulated symbols. Transmit power: $\measPWRbad \si{dBm}$}
  \label{fig:bad_diagrams} 
\end{figure}

% !TEX root = main.tex
\begin{table}[htbp]
  \centering
  \caption{Measured performance parameters. Transmit power: $\measPWR \si{dBm}$}
    \begin{tabular}{lc}
    \rowcolor[rgb]{ 0,  0,  0} \textcolor[rgb]{ 1,  1,  1}{\textbf{System Parameter}}	& \textcolor[rgb]{ 1,  1,  1}{\textbf{Measured Value}} 		\\
    \rowcolor[rgb]{ 0,  0,  0} \textcolor[rgb]{ 1,  1,  1}{} & \textcolor[rgb]{ 1,  1,  1}{\textbf{Modulation QPSK / QAM-16}}					\\

    	SNR														& $\measSNRGood\si{dB}$						\\
    	EVM 													& $\measEVMGood \si{dB}$						\\
    	BER			 											& \measBERQPSKGood / \measBERQAMGood		\\
 \end{tabular}
  \label{tab:meas_params_good}
\end{table}
% !TEX root = main.tex
\begin{table}[htbp]
  \centering
  \caption{Measured performance parameters. Transmit power: $\SI{\measPWRbad}{dBm}$}
    \begin{tabular}{lc}
    \rowcolor[rgb]{ 0,  0,  0} \textcolor[rgb]{ 1,  1,  1}{\textbf{System Parameter}}	& \textcolor[rgb]{ 1,  1,  1}{\textbf{Measured Value}} 		\\
    \rowcolor[rgb]{ 0,  0,  0} \textcolor[rgb]{ 1,  1,  1}{} & \textcolor[rgb]{ 1,  1,  1}{\textbf{Modulation QPSK / QAM-16}}					\\

    	SNR 													& $\SI{\measSNRQPSKBad}{dB} / \SI{\measSNRQAMBad}{dB}$					\\
	\ebnot 													& $\SI{11.9}{dB} / \SI{3.8}{dB}$								\\    	
	BER			 											& \measBERQPSKBad / \measBERQAMBad		\\
	EVM 													& $\SI{-16.1}{dB} / \SI{-12.8}{dB}$								\\

 \end{tabular}
  \label{tab:meas_params_bad}
\end{table}

Figure \ref{fig:ber_vs_snr} shows the measured BER as a function of estimated \ebnot. This value is compared to a theoretical BER (Estimated BER) to give an impression of the system performance under the given conditions.
% !TEX root = main.tex
\begin{figure} 
    \centering
  \subfloat[QPSK \label{1a}]{%
\includegraphics[width=0.45\linewidth]{BERvsSNRQPSK.png}
}
  \subfloat[QAM-16\label{1b}]{%
\includegraphics[width=0.45\linewidth]{BERvsSNRQAM.png}

}
  \caption{BER vs. SNR for both modulation formats. The estimated BER is the theoretical BER based on the estimated values for \ebnot and true BER is the actual calculated BER}
  \label{fig:ber_vs_snr} 
\end{figure}

\input{QAMBarker}

\subsection{Discussion of Obtained Results}
% Brief description of overall system performance
The proposed system is supposed to broadcast speech data with adaptable sound quality, by switching between QPSK and QAM-16 modulation based on the detected error rates. During the full system test, the threshold for changing state was set to \BERThreshold. With this threshold, the system changed quality state when the transmit power was reduced to about $\BERPowerLimit$ dBm. The test verified that the adaptable quality feature works as expected. The measured $\measDelay \si{ms}$ delay show that the system is well suited for two-way communication as well as broadcasting. 

% Short comment on power spectrum
Figure \ref{fig:pwr_spectrum} shows that the measured power spectrum lies very close to the theoretical spectrum. This indicates that the amount of nonlinear distortion introduced by the RF hardware is at a minimum.

% Short qualitative description of behaviour under QPSK modulation
When transmitting QPSK modulated data at low data rate, the system delivered almost noise free sound, at both power levels. The sound quality was not audibly reduced by the transmission. Figure \ref{fig:good_diagrams} and \ref{fig:bad_diagrams} and table \ref{tab:meas_params_good} and \ref{tab:meas_params_bad} supports this qualitative description. 

% Discussion of QAM-16 results
The transmission of QAM-16 modulated data was not equally successful. At the highest power level, the transmitted speech was barely audible. At the lowest power level the sound was completely lost in noise. This result is again supported by figure \ref{fig:good_diagrams} and \ref{fig:bad_diagrams} and table \ref{tab:meas_params_good} and \ref{tab:meas_params_bad}. The constellation diagram in figure \ref{fig:good_diagrams} (b) shows that the poor quality is mainly due to bad phase synchronization. Figure \ref{fig:QAMBarker} shows a comparison of the QAM-16 constellation with all frame symbols (a), and with the Barker symbols only (b). As the figure shows, the Barker symbols have no visible phase shift, which indicates that there is a phase difference between the Barker symbols and the rest of the frame. The reason for this problem is not understood by the authors and the issue remains for future improvements.

% Comment on BER vs Eb/N0 plot
ADD COMMENT ON BER VS Eb/N0


% CONCLUSION
% !TEX root = main.tex
\section{Conclusion}
\label{sec:conclusion}
The design and implementation of a radio communication system for broadcasting of speech with adaptable data rate is presented. The system adapts the transmitted data rate to the state of the radio channel by evaluating detected bit error at the receiver. The transmitted data rate is varied by a factor 2 by switching modulation format between QPSK and QAM-16. 

The system was verified in an indoor environment with a transmission distance of 1 meter and variations in the radio channel is simulated by adjusting the transmit power. The system transmit at high data rate (\SI{520,8}{kb/s}) when using QAM-16 modulated symbols at \SI{\measPWR}{dBm} and low data rate (\SI{249}{kb/s}) for QPSK at \SI{\measPWRbad}{dBm}. Bit error rates of \measBERQAMGood and \measBERQPSKBad was measured at high and low data rate respectively.

The high BER for QAM-16 modulation makes the effect of adaptable quality hardly audible because of the high noise level when transmitting at high data rate. The adaptable quality feature is however interesting in itself and this demonstration shows a working ``proof of concept''. By adapting the transmitted data rate to the state of the radio channel, the system yields a better utilization of available resources. This two-level adaption could be extended to several levels for higher performance in future improvements.

% BIBLIOGRAPHY
\bibliographystyle{IEEEtran}
\bibliography{references}

% APPENDICES
\appendices

\section{Block Diagram}
\label{a:block_diagram}
% !TEX root = main.tex
\begin{figure*}[htbp]
\centering
\begin{tikzpicture}[                
                    box/.style={
            		draw,
			thick,
            		text centered,
            		minimum width=1cm,
            		minimum height=1cm,
			font=\scriptsize,
			align=center,
			anchor=center,
            	}, decision/.style={
			draw,
			diamond,
			thick,
			align=center,
            		minimum width=1cm,
            		minimum height=1cm,
			font=\scriptsize,
			anchor=center,	
		}, ]
	
% ------ USRP ------- %
\filldraw[fill=black!20!white, draw=black!20!white, opacity=0.5] (13, 2.5) rectangle (15,-6.5);
\draw (14, 2.5) node[anchor=north, align=center]{RF \\ Hardware};

% -------- TX --------- %
\filldraw[fill=black!10!white, draw=black!10!white, opacity=0.5] (-2.5,2.5) rectangle (13,-1);
\draw (-2.5,2.5) node[anchor=north west]{TX - Software};
% Draw boxes
\draw 
(6, 1.5) node[box, minimum width=7cm, font=\footnotesize](state){Session State}

(0,0) node[box](sound_prod){Sound \\ producer} 
(2,0) node[box](source_enc){Source \\ encoder} 
(4,0) node[box](packing_tx){Packing} 
(6,0) node[box](fec_tx){FEC} 
(8,0) node[box](symbol_mapping_tx){Symbol \\ mapping} 
(10,0) node[box](barker_tx){Add \\ Barker} 
(12,0) node[box](pulse_shape_tx){Pulse \\ shaping} 
(14,0) node[box](usrp_tx){USRP \\ TX} 
;


\draw [->, thick] (sound_prod.east) -- node[above, font=\tiny, align=center]{\rawDataRate} node[below, font=\tiny, align=center]{Mb/s} (source_enc.west);
\draw [->, thick] (source_enc.east) -- node[above, font=\tiny, align=center]{\sourceDataRateQPSK /  \\ \sourceDataRateQAM} node[below, font=\tiny, align=center]{kb/s} (packing_tx.west);
\draw [->, thick] (packing_tx.east) -- node[above, font=\tiny, align=center]{\packetDataRateQPSK /  \\ \packetDataRateQAM} node[below, font=\tiny, align=center]{kb/s} (fec_tx.west);
\draw [->, thick] (fec_tx.east) -- node[above, font=\tiny, align=center]{\fecDataRateQPSK /  \\ \fecDataRateQAM} node[below, font=\tiny, align=center]{kb/s}  (symbol_mapping_tx.west);
\draw [->, thick] (symbol_mapping_tx.east) -- node[above, font=\tiny, align=center]{\symbolMapRateQPSK /  \\ \symbolMapRateQAM} node[below, font=\tiny, align=center]{ksymb/s}  (barker_tx.west);
\draw [->, thick] (barker_tx.east) -- node[above, font=\tiny, align=center]{\barkerRateQPSK /  \\ \barkerRateQAM} node[below, font=\tiny, align=center]{ksymb/s} (pulse_shape_tx.west);
\draw [->, thick] (pulse_shape_tx.east) -- (usrp_tx.west);

% State lines
\draw [->, thick] (state.south) ++(-3, 0) -- ++(0, -0.2) -| (source_enc.north);
\draw [<-, thick] (packing_tx.north) -- ++(0, 0.5); 
\draw [<-, thick] (symbol_mapping_tx.north) -- ++(0, 0.5); 
\draw [->, thick] (state.south) ++(3, 0) -- ++(0, -0.2) -| (barker_tx.north);


% -------- RX --------- %
\filldraw[fill=black!10!white, draw=black!10!white, opacity=0.5] (-2.5,-1.5) rectangle (13,-6.5);
\draw (-2.5,-1.5) node[anchor=north west]{RX - Software};
% Draw boxes
\draw 
(-1.5,-3) node[box](sound_cons){Sound \\ consumer} 
(0,-3) node[box](source_decode){Source \\ decoder} 
(1.5,-3) node[box](unpack_rx){Unpacking} 
(3,-3) node[box](fec_rx){FEC \\ decoding} 
(4.5,-3) node[box](demap_rx){De- \\ mapping} 
(6,-3) node[box](freq_sync_rx){Frequency \\ and \\ phase \\ sync} 
(7.5,-3) node[box](frame_sync_rx){Frame \\ sync} 
(9,-3) node[box](bfs_rx){Blind \\ frequency \\ search} 
(10.5,-3) node[box](symb_sync_rx){Symbol \\ sync} 
(12,-3) node[box](filter_rx){Matched \\ filter} 
(14,-3) node[box](usrp_rx){USRP \\ RX} 
;

\node [decision, below of=frame_sync_rx, yshift=-1cm] (barker_length_dec) {Barker \\ length?};
\node [box, minimum height=0cm, left of=barker_length_dec, yshift=0.5cm, xshift=-0.5cm](qpsk_out){Modulation \\ QPSK};
\node [box, minimum height=0cm, left of=barker_length_dec, yshift=-0.5cm, xshift=-0.5cm](qam_out){Modulation \\ QAM-64};
\node [box, below of=fec_rx, yshift=-1cm,](send_error){Send BER \\ to \\ transmitter};

\draw [->, thick] (usrp_tx.south) -- ++ (-0.1, -1) -- ++(0.2, 0.2) node[right, align=center, font=\scriptsize, anchor=west, xshift=-0.1cm]{Speech \\ data} -- (usrp_rx.north);
\draw [->, thick] (usrp_rx.west) -- (filter_rx.east);
\draw [->, thick] (filter_rx.west) -- (symb_sync_rx.east);
\draw [->, thick] (symb_sync_rx.west) -- (bfs_rx.east);
\draw [->, thick] (bfs_rx.west) -- (frame_sync_rx.east);
\draw [->, thick] (frame_sync_rx.west) -- (freq_sync_rx.east);
\draw [->, thick] (freq_sync_rx.west) -- (demap_rx.east);
\draw [->, thick] (demap_rx.west) -- (fec_rx.east);
\draw [->, thick] (fec_rx.west) -- (unpack_rx.east);
\draw [->, thick] (unpack_rx.west) -- (source_decode.east);
\draw [->, thick] (source_decode.west) -- (sound_cons.east);

\coordinate [below of=demap_rx, xshift=0.2cm, yshift=0.5cm](qpsk_in) {};
\coordinate [below of=demap_rx, xshift=-0.2cm, yshift=0.5cm](qam_in) {};

\draw [->, thick] (frame_sync_rx.south) -- node[right, font=\scriptsize, align=center] {Detected \\ Barker} (barker_length_dec.north);
\draw [->, thick] (barker_length_dec.west) ++(0.5,0.5) -- node[above, font=\scriptsize]{7} (qpsk_out.east);
\draw [->, thick] (barker_length_dec.west) ++(0.5,-0.5) -- node[below, font=\scriptsize]{13} (qam_out.east);
\draw [->, thick] (qpsk_out.west) -| (qpsk_in);
\draw [->, thick] (qam_out.west)  -| (qam_in);
\draw [->, thick] (fec_rx.south)  -- node[left, font=\scriptsize, align=center]{Number of \\ detected \\ errors} (send_error.north);


\end{tikzpicture}

\caption{Block diagram of data packet system. This block diagram shows the forward path of the system, where speech data is being transmitted.}
\label{fig:block_diagram}
\end{figure*}
% !TEX root = main.tex
\begin{figure*}[htbp]
\centering
\begin{tikzpicture}[                
                    box/.style={
            		draw,
			thick,
            		text centered,
            		minimum width=1cm,
            		minimum height=1cm,
			font=\scriptsize,
			align=center,
			anchor=center,
            	}, decision/.style={
			draw,
			diamond,
			thick,
			align=center,
            		minimum width=1cm,
            		minimum height=1cm,
			font=\scriptsize,
			anchor=center,	
		}, ]
	
% ------ USRP ------- %
\filldraw[fill=black!20!white, draw=black!20!white, opacity=0.5] (13, 1.5) rectangle (15,-6.5);
\draw (14, 1.5) node[anchor=north, align=center]{RF \\ Hardware};

% -------- TX --------- %
\filldraw[fill=black!10!white, draw=black!10!white, opacity=0.5] (-1.5,1.5) rectangle (13,-1);
\draw (-1.5,1.5) node[anchor=north west]{TX - Software};
% Draw boxes
\draw 
(4,0) node[](detected_errors){} 
(6,0) node[box](packing_tx){Packing} 
(8,0) node[box](symbol_mapping_tx){Symbol \\ mapping} 
(10,0) node[box](barker_tx){Add \\ Barker} 
(12,0) node[box](pulse_shape_tx){Pulse \\ shaping} 
(14,0) node[box](usrp_tx){USRP \\ TX} 
;

\draw [->, thick] (detected_errors.east) -- node[above, font=\footnotesize, align=center]{Detected \\ Errors} (packing_tx.west);
\draw [->, thick] (packing_tx.east) -- (symbol_mapping_tx);
\draw [->, thick] (symbol_mapping_tx.east) --  (barker_tx.west);
\draw [->, thick] (barker_tx.east) -- (pulse_shape_tx.west);
\draw [->, thick] (pulse_shape_tx.east) -- (usrp_tx.west);


% -------- RX --------- %
\filldraw[fill=black!10!white, draw=black!10!white, opacity=0.5] (-1.5,-1.5) rectangle (13,-6.5);
\draw (-1.5,-1.5) node[anchor=north west]{RX - Software};
% Draw boxes
\draw
(3,-3) node[box](unpack_rx){Unpacking} 
(4.5,-3) node[box](demap_rx){De- \\ mapping} 
(6,-3) node[box](freq_sync_rx){Frequency \\ and \\ phase \\ sync} 
(7.5,-3) node[box](frame_sync_rx){Frame \\ sync} 
(9,-3) node[box](bfs_rx){Blind \\ frequency \\ search} 
(10.5,-3) node[box](symb_sync_rx){Symbol \\ sync} 
(12,-3) node[box](filter_rx){Matched \\ filter} 
(14,-3) node[box](usrp_rx){USRP \\ RX} 
;

\draw [->, thick] (usrp_tx.south) -- ++ (-0.1, -1) -- ++(0.2, 0.2) node[right, align=center, font=\scriptsize, anchor=west, xshift=-0.2cm, yshift=0.1cm]{Detected \\ errors} -- (usrp_rx.north);\draw [->, thick] (usrp_rx.west) -- (filter_rx.east);
\draw [->, thick] (filter_rx.west) -- (symb_sync_rx.east);
\draw [->, thick] (symb_sync_rx.west) -- (bfs_rx.east);
\draw [->, thick] (bfs_rx.west) -- (frame_sync_rx.east);
\draw [->, thick] (frame_sync_rx.west) -- (freq_sync_rx.east);
\draw [->, thick] (freq_sync_rx.west) -- (demap_rx.east);
\draw [->, thick] (demap_rx.west) -- (unpack_rx.east);

\draw[->, thick] (demap_rx.south) ++(1, -0.7) -- node[above, font=\scriptsize]{QPSK} ++(-1, 0) -- (demap_rx.south);

\node [decision, left of=unpack_rx, xshift=-0.8cm] (change_state) {Change \\ state?};
\node [box, below of=change_state, yshift=-0.5cm, minimum height=0] (do_nothing) {Do nothing};
\node [box, left of=change_state, xshift=-0.8cm, minimum height=0] (do_change) {Change \\ state};

\draw [->, thick] (unpack_rx.west) -- (change_state.east);
\draw [->, thick] (change_state.south) -- node[right, font=\scriptsize, yshift=3]{No} (do_nothing.north);
\draw [->, thick] (change_state.west) -- node[above, font=\scriptsize]{Yes} (do_change.east);


\end{tikzpicture}

\caption{Block diagram of BER packet system. This sub-system constitutes the feedback path where the receiver transmit information about detected error rate back to the transmitter.}
\label{fig:block_diagram_feedback}
\end{figure*}

\end{document}