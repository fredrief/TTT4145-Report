% !TEX root = main.tex
\section{Link Budget}
\label{sec:link_budget}
The link budget for the system is shown in table \ref{tab:link_budget}. As the purpose of the system is to demonstrate the feedback feature, the system is designed for the test environment only, which is reflected in the link budget. The system is designed to operate indoors with a distance of 10 meters between transmitter and receiver. Table \ref{tab:link_budget} shows losses from path loss, loss in TX and RX and some estimated key parameters at the receiver, such as $E_b/N_0$ and BER. 

As the system is designed to switch between two different modulation formats the received \ebnot should not be carefully tuned. The link budget is designed such that the \ebnot is mostly good enough for QAM-64 modulation under line of sight (LOS), but forces the system to switch to QPSK if the LOS is lost. 

The different parts of the link budget will be discussed in this section.

\subsection{TX and RX Loss}
\label{sec:txandrxloss}
The value for connector loss is taken from datasheets of standard coaxial RF connectors \cite{rfconnector}. The antenna gain value is taken from the data sheet \cite{antenna} which reports a peak gain of 3.4 dBi. We used the value 3 dBi in the link budget to account for suboptimal conditions. The launch power, PA Power, was adjusted after measurements to obtain appropriate $E_b/N_0$ at the receiver.

\subsection{Path Loss}
\label{sec:path_loss}
The estimated path loss constitutes solely of the propagation loss obtained from the ITU Indoor Propagations Loss Model \cite{itu_model}. The loss model consists of two adjustable factors, the distance power loss coefficient, $N$, and the floor loss penetration factor, $P_f(n)$. The latter is set to 0, and the former was set to 38 after calibrating the test environment. Other loss factors such as pointing loss and polarisation loss was considered but measurements showed that the amount of reflections in the room made pointing and polarisation irrelevant to the received power. More details on the measurements is given in section \ref{sec:verification}. 

  
\subsection{RX Noise}
\label{sec:rxnoise}
The antenna noise density was measured with a spectrum analyser and the estimated value was taken as an average of several single runs. The noise figure of the receiver is included to account for noise added by the radio hardware, with value taken from the data sheet. The small scale fading margin, $M_{ssf}$, is included to account for variations in received power. This margin was obtained by evaluating several measurements of received power using a spectrum analyser. The particular value is taken to be two times the standard deviation of the measured values. 

\subsection{RX Properties}
\label{sec:rxproperties}
Some key properties of the received signal is calculated based on the estimated values in the link budget. The BER and $E_b/N_0$ is calculated for the two modulation schemes separately. The bit error rate is calculated for QPSK and QAM-64 by equation \ref{eq:berqpsk} and \ref{eq:berqam} respectively.

\begin{equation}
\label{eq:berqpsk}
P_B \approx \frac{1}{2}\erfc\sqrt{\frac{E_b}{N_0}}
\end{equation}

\begin{equation}
\label{eq:berqam}
P_B \approx \frac{2}{\log_2M}\left(1-\frac{1}{M}\right)\erfc\left(\sqrt{\frac{3\log_2M}{2(M-1)}\cdot \frac{E_b}{N_0}}\right)
\end{equation}


